\documentclass{article}
\usepackage{multicol}
\usepackage{braket}
\usepackage{geometry}
\usepackage{setspace}
\usepackage{lipsum}
\usepackage{amssymb}
\usepackage{tikz-cd}
\usepackage{amsthm}
\usepackage{amsmath}
\usepackage{tabularx}
\geometry{top=1in,bottom=1in}



\title{Algorithms Homework Assignment 2}
\author{Andrew Osborne}

\newcommand{\nlg}[3]{n^{\log_{#2}(#1) \, #3}}
\begin{document}
  \maketitle

  \section*{Conventions}
  When I refer to $\mathbb{N}$, I speak of $$\mathbb{N} \, = \, \{ 1\,,2\,,3\,,\,\dots \}$$
  And, when I label a variable to be $m$ or $n$, I am indicating that these variables take values only in $\mathbb{N}$.

  \section*{Problem 4.5-1}
    \subsection*{Part a.}

      Our recursion equation is $$T(n) = 2T(n/4) + 1$$
      Then, in the context of the master theorem, we have
      $ a = 2, \quad b = 4,\quad f(n) = 1 $.
      Then, we can see clearly that $\log_{4}(2) = \frac{1}{2}$.
      If $\epsilon = \frac{1}{4}$, we see that $n^{\log_4{2} - \epsilon} = n^{\frac{1}{2} - \frac{1}{4}} = n^{\frac{1}{4}}$ and $ 0 \leq 1 \leq n^{\frac{1}{4}} \quad \forall n \geq 1$ so clearly
      $f(n) = O(n^{\frac{1}{4}})$ and $$T(n) = \Theta(n^{\frac{1}{2}})$$

    \subsection*{Part b.}
      Our recursion equation is $$ T(n) = 2T(n/4) + \sqrt{n}$$
      Then, from the last problem, we know that $n^{\log_b(a)} = \sqrt{n}$ and,
      in this case, $f(n) = \sqrt{n}$ so clearly
      $$ \sqrt{n} = \Theta(\sqrt{n}) $$
      and therefore, by case 2 of the master theorem, $$ T(n) = \Theta(\sqrt{n} \log_{2}(n)) $$

    \subsection*{Part c.}
      Our recursion euation is $$T(n) = 2T(n/4) + n $$
      Once again, from our previous work, we know that $n^{\log_b(a)} = \sqrt{n}$.
      Then take $\epsilon = \frac{1}{4}$ and $$n^{\frac{1}{2} + \epsilon} = n^{\frac{3}{4}} $$
      Then, for $ c =1$ and $n_0 = 1$,
      $$ 0 \leq n^{\frac{3}{4}} \leq n \quad \forall \,\, n \geq n_0$$
      so $f(n) = \Omega(n^{\log_b(a) + \epsilon})$ for some constant $\epsilon > 0 $ and $ a\,f(n/b) = \frac{a}{b} \, n = \frac{n}{2}$.
      and clearly, if $c = \frac{3}{4}$, then
      $$ \frac{1}{2} n \leq \frac{3}{4}n \quad \forall \,\, n \in \mathbb{N} $$
      Therefore, by the third case of the master theorem,
      $$T(n) = \Theta(n) $$

    \subsection*{Part d.}
      Our recursion relation is the same as the above three but with $f(n) = n^2$.
      And, again, $n^{\log_b\,a} = \sqrt{n}$ and clearly with $\epsilon = 1$,
      $\nlg{a}{b}{+\epsilon} = n^{\frac{3}{2}}$.
      Furthermore, for all $ n \geq 2$, $$0 \leq n^{\frac{3}{2}} \leq n^2 $$
      So $f(n) = \Omega(\nlg{a}{b}{+\epsilon})$.
      Additionally, we can see that $a \, f(n/b) = a \, \frac{n^2}{b^2} = \frac{1}{8} n^2 \leq \frac{n^2}{2} \quad \forall \,\, n \in \mathbb{N}$.
      Then, by the third case of the master theorem, $$T(n) = \Theta(n^2)$$

  \section*{Problem 4-1}
    \subsection*{Part a.}
      Our recursion equation is $$T(n) = 2 T(n/2) + n^4$$
      We see that $\nlg{a}{b}{} = n$.
      Then quite clearly, if we take $\epsilon = 1$ then $n^4 = \Omega(\nlg{2}{2}{+1}) = \Omega(n^2)$.
      Moreover, $a \, f(n/b) = \frac{1}{8} n^4 \leq \frac{1}{2} n^4 \quad \forall n \in \mathbb{N}$.
      Therefore by the master theorem, $$T(n) = \Theta(n^4)$$

    \subsection*{Part b.}
      Our recursion relation is $$ T(n) = T(7n/10) + n$$
      Then $\nlg{a}{b}{} = n^0 = 1$.
      And if we let $\epsilon = \frac{1}{2}$, then $\nlg{a}{b}{+\epsilon} = \sqrt{n}$ and $$ \forall \,\, n \in \mathbb{N}\,,\, 0 \leq \sqrt{n} \leq n$$
      So then $n = \Omega(\nlg{a}{b}{+\epsilon}) = \Omega(\sqrt{n})$.
      %Quite clearly $n = \Omega(1)$ because $$ 0 \leq 1 \leq n \quad \forall \,\, n \in \mathbb{N}$$
      Then $a \, f(n/b) = \frac{7}{10} n \leq \frac{8}{10} n \quad \forall \,\, n \in \mathbb{N}$.
      Therefore, by the master theorem, $$T(n) = \Theta(n)$$.

    \subsection*{Part c.}
      Our recursion relation is $$T(n) = 16 T(n/4) +  n^2$$
      Then $\nlg{a}{b}{} = n^2 $ and it is obvious that $n^2 = \Theta(n^2)$ so by the master theorem, $$T(n) = \Theta(n^2 \log_2 n)$$

    \subsection*{Part d.}
      Our recursion relation is $$T(n) = 7 T(n/3) + n^2 $$
      Then $\nlg{a}{b}{} = \nlg{7}{3}{}$.
      Note that $ 1 < \log_{3}(7) < 2$.
      In fact $\log_{3}(7) = 1.77124\dots$.
      Then, for $\epsilon = 0.1$, $\nlg{7}{3}{+0.1} = n^{1.87124\dots}$ and for all $n \in \mathbb{N}$, $ 0 \leq n^{1.87124\dots} \leq n^2 $ so
      $ n^2 = \Omega(n^{1.87124\dots})$ and $a \, f(n/b) = \frac{7}{9} n^2 \leq \frac{8}{9} n^2 \quad \forall \,\, n \in \mathbb{N}$, so, by the master theorem, $$T(n) = \Theta(n^2)$$

    \subsection*{Part e.}
      Our recursion relation is $$ T(n) = 7 T(n/2) + n^2 $$
      Then, $\nlg{a}{b}{} = \nlg{7}{2}{} \in (2,3)$.
      In fact, $ \nlg{7}{2}{} = n^{2.807\dots}$.
      Then if $\epsilon = 0.1$, then \\$\nlg{7}{2}{-0.1} = n^{2.707\dots}$.
      Finally, using $ c= 1$ and noticing that $ 0 \leq n^2 \leq n^{2.707\dots} \quad \forall \,\, n \in \mathbb{N}$, we see that $n^2 = O(\nlg{a}{b}{-\epsilon})$
      and therefore that $$T(n) = \Theta(n^{2.807\dots})$$

    \subsection*{Part f.}
      Our recursion relation is $$ T(n) = 2T(n/4) + \sqrt{n}$$
      Then $\nlg{a}{b}{} = \sqrt{n}$.
      Obviously $\sqrt{n} = \Theta(\sqrt{n})$ and therefore by the master theorem $$T(n) = \Theta(\sqrt{n} \,\,lg\, n)$$
      I solved this problem exactly earlier in the homework.

    \subsection*{Part g.}
      Our recursion relation is $$T(n) = T(n-2) + n^2$$
      We cannot apply the master theorem to this situation because of the form of the recursion equation.
      Assuming that $n$ is even for the moment, then $T(n) = n^2 + (n-2)^2 + (n-4)^2 + \dots +(4)^2+ T(2)$.
      This is precisely $$ T(2) + \sum_{i=2}^{n/2} (2\,i)^2 = T(2) - 4 + \sum_{i=0}^{n/2}(2i)^2  $$
      but, noting our knowledge of series', this is
      $$T(n) =  T(2) - 4 + \frac{1}{6}(2(n/2)^3  + 3(n/2)^2 + n/2)$$
      We note that, from this expression, we may only claim formally that
      $T(n) = \Omega(n) \,\, \& \,\, T(n) = O(n^3)$.
      If we tolerate some sloppiness, however, and consider only the term of the greatest asymptotic growth, we can state that $$T(n) = \Theta(n^3)$$
      We have here assumed $n$ to be even, but the same argument may be applied directly to
      $$ T(1) - \sum_{i=2}^{(n-1)/2}(2i+1)^2$$
      to acquire exactly the same result.
      Furthermore, we may have guessed correctly that a sum of squares is cubic in leading order, just as the integral of a quadratic is cubic in growth.
      My tolerance for sloppy analysis is justified in section 4.6 in the text.

  \section*{References}

    I used this website to look--up some series identities that were used in  solving recursion equations
    \begin{verbatim}
      https://en.wikipedia.org/wiki/List_of_mathematical_series
    \end{verbatim}



\end{document}
