\documentclass{article}

\usepackage{geometry}
\geometry{top=0.5in,bottom=0.5in}

\usepackage{amssymb}
\newcommand{\N}{\mathbb{N}}

\author{Andrew Osborne}
\title{Algorithms Homework 7}

\begin{document}
 % \maketitle  
  \begin{center}
    \textbf{Algorithms Homework 7: Andrew Osborne}
  \end{center} 

    
    As noted in the text, the activity problem exhibits an optimal structure 
    which facilitates a dynamic programming approach, and my algorithm 
    uses memoization to achieve a polynomial runtime in the modified 
    activity selection problem.
  \section*{Data Structures}

    I use an object-oriented structure to store activities rather than some 
    combination of arrays which are used in parallel. 
    Each activity, it's start time, it's ID, it's finish time, and it's value 
    are stored in a single object. 
    The primary reason for this is to make sorting more convenient; more on that 
    later.
    Yet another class is used to store an array of activities in addition to the 
    interval information and total number of activities which are provided in the 
    input file. 
    I also add two ``ficticious" activities to facilitate the solution of our problem.
    One of these activities ends at time $0$ and the other begins at the end
    of the specified time interval.
    By doing this, we can restrict ourselves to the problem of finding
    the optimal schedule of activities between two activities with no loss
    of generality. 
    If we did not have lower and upper bounds for our time interval,
    we would be unable to use this without additional preprocessing.

  \section*{Algorithm}
    For the remainder of this section, I will use $f_i$ to denote the finish
    time of the activity with ID $i$, and I will use $s_i$ to denote the start
    time of the activity with ID $i$.
    In my analysis, I will assume that ID's are assigned in order of nondecreasing
    finish time, but in my code, I remap ID's to thier i-th order statistic in
    order of increasing finish time.
    That is, I will here pretend that, at the time the algorithm begins, the 
    activities have already been sorted in order of increasing finish time.
    In my code, I use insertion sort ($\Theta(n^2)$ worst--case runtime)
    to sort activities, but the object oriented structure above preserves
    activity IDs so that reporting activity values in the way that they were 
    provided is trivial.
    Furthermore, I will refer to the activity with ID $i$ as $a_i$.
    I will also use $c_{ij}$ to refer to the value of an optimal schedule which 
    takes place between $f_i$ and $s_j$.

    We may think of the un--modified activity selection problem as a special case of
    our activity problem wherein the value of each activity is $1$, or any constant
    value for that matter. 
    Furthermore, we are concerned with finding the optimal value of a schedule between
    any two activities, and we want to be able to reconstruct the schedule corresponding to any such value.
    I present the following pseudocode for my algorithm.
    \begin{verbatim}
    // assuming each activity has .id, .start, .end, .value properties
    read(file)
      let A[0..n] be a new array of activities

      let A.n = file.number_of_activities + 2
      let A.u = file.max_time
      
      let A[0].finish = 0
      let A[n].start = u

      for i = 1 to n-1:
        let A[i] = file.ith_activity
      return A

    solve(A)
      let c[0..A.n,0..A.n] be a new array of values
      let p[0..A.n,0..A.n] be a new array of integers
      let flags[0..A.n] be a new array of booleans

      for i = 2 to A.n-1 // O(n)
        for j = 0 to A.n - i // O(n - i)
          k = j + i
          l = k - 1
          
          while A[j].finish < A[l].finish // will terminate at or before l = j
            // O(k - j) = O(i) worst case            
            newValue = A[l].value + c[j,l] + c[l,k]
            if A[j].finish <= A[l].start and A[l].finish <= A[k].start
              if newValue > c[j,k] and 
                c[j,k] = newValue 
                p[j,k] = l
                // if we just beat the optimal solution, there are no 
                // other solutions which have been found which are optimal
                flags[everywhere] = false
              else if newValue == c[j,k] and j == 0 and k == n
                // if we are on the last iteration and we have a duplicate
                // optimal value, we may have found a nonunique solution
                flags[l] = true
            l = l-1
      return flags, c, p

    \end{verbatim}
    We are above checking for the optimal value of the schedule of events in order of 
    increasing width. 
    That, is, we check between $a_0$ and $a_2$ (noting that only $a_1$ may fill this 
    space becasue of the sorting of activities), and then we chech in between $a_1$ 
    and $a_3$ and so on.
    We continue in this way, checking all schedules of width $i$ each of which 
    consist of at most $i-1$ checks and of which there are $n -i$ so that, our total runtime, summing over i, is:
    \[ \sum_{i=2}^{n}i\,n - i^2 = O(n^3)\] in the worst case.
    Then, after our main function runs, our optimal value will be stored in 
    \verb c[0,n]  and our optimal schedule can be ascertained by calling 
    \begin{verbatim}
      schedule(p,i,j)
        print p[i,j]
        schedule(p,i,p[i,j])
        schedule(p,p[i,j])

      schedule(p,0,n)
    \end{verbatim}
    And, from our earlier function, if any of the elements of \verb flags[i]  is
    true and \verb i  is not in the schedule reported by \verb schedule(p,0,n)  then
    our solution is not unique. 
    Otherwise, if the only values of \verb flags  which are true are elements
    of the reported schedule, then our solution is unique. 
    Checking this can be done in $O(n)$ time. 
    In my code, things are a bit more complicated because I wanted to recover
    all optimal schedules, which can also be done in $O(n)$ time, but which requires
    some more involved code.

  \section*{Testing}
    To test this code, I shuffled the provided input files to verify that my code 
    gives an answer which agrees with my inspection, and I then wrote a python 
    script which generates random input files in which ID values and finish 
    times are shuffled randomly.
    I shuffled lists of length 100000 1000 times and verified that my answers agreed on all trials. 
    My code is correct and complete.



\end{document}
