\documentclass{article}
\usepackage{amsmath}
\usepackage{amsthm}
\usepackage{geometry}
\geometry{top=1in,bottom=1in}

\author{Andrew Osborne }
\title{Algorithms Assignment 6}

\begin{document}
  \maketitle

  \section*{14.1-3}
  \begin{verbatim}
    OS-Select(x,i) // finds ith OS in tree rooted by x
      y = x // dummy variables
      j = i
      while j != y.left.size+1:

        if j < y.left.size+1
          y = y.left

        elseif j > y.left.size+1:
          j = j - y.left.size - 1
          y = y.right

      return y

  \end{verbatim}

  \section*{14.1-5}
    This routine will find the ith successor of a node x in an $n$ element
    order--statistic tree, $T$.

    \begin{verbatim}
      Ith-Successor(T,x,i)
        j = OS-RANK(T,x) // O(lg n) runtime
        return OS-Select(T.root,j+i) // O(lg n) runtime
    \end{verbatim}
    It is clear that the routine posed above runs in $c_{1}lg\,(n) + c_{2}\, lg(n) = (c_1 + c_2)\, lg\,\,n = O(lg\,\,n)$ time in the worst case, so this is satisfactory for our purposes.

  \section*{14.1-6}
    We must write some subroutines to modify existing logic for the purpose of this problem.

    \begin{verbatim}
      // to be called AFTER a right rotation
      Right-Rotate-Fix(x)
        x.rank = x.rank - x.p.rank

      // to be called AFTER a left rotation
      Left-Rotate-Fix(x)
        x.p.rank = x.rank + x.p.rank

    \end{verbatim}
    These two routines will be used to be used to preserve rank--information under rotations.
    That is, when preforming a left--rotation on a node, say x, then x becomes the left child of x.right, whose left subtree is transplanted onto the right subtree of x. In this case, the rank of x will not be altered under a left rotation, but (x.p after rotation or x.right before rotation) needs to have its rank increased by the rank of x.
    On the other hand, under right--rotation, the rank of x will change, because of the replacement of the left subtree of x with the right subtree of x.left (before rotation).

    That may be unclear, but we may simply append \verb Right-Rotate-Fix(x)  to the last line of \verb RIGHT-ROTATE,   and we may append \verb Left-Rotate-Fix(x)  to the last line of \verb LEFT-ROTATE  in order to prevent rotations from damaging our rank information.

    On the other hand, at some point during the insertion process, any node to be inserted will be a leaf node in some tree.
    At this point, after the initial binary search tree insertion, before calling any FIXUP functions, we may call
    \begin{verbatim}
      Increment-Insert(x)
        if x.p != NIL
          if x == x.p.left
            x.p.rank = x.p.rank + 1
          Increment-Insert(x.p)
    \end{verbatim}
    on the inserted node in order to maintain rank information during insertion, while the rotation operations above maintain rank information during the fixup process which comes after insertion.

    During deletion, we need only reduce the rank of elements between the element to be deleted and the root of our tree.
    \begin{verbatim}
      Decrement-Delete(x)
        if x.p != NIL
          if x == x.p.left
            x.p.rank = x.p.rank 1 1
          Increment-Insert(x.p)
    \end{verbatim}
    That is, before deleting the specified element from the tree, call \verb Decrement-Delete(x)  on the element to be deleted.


  \section*{14.1-7}
    Consider an array of real valued constants, $A = [a_1, a_2, a_3, \dots , a_n]$.
    Then for $1 \leq k \leq n$, let $r_k$ be the rank of $a_k$ in $A$, and let
    $R = [r_1, r_2, r_3, \dots , r_n]$.
    Using one's intuition it is clear that $r_1$ is the rank of $a_1$ and also the number of elements beyond $a_1$ which are less than $a_1$ minus one!
    That is, there are $r_1 -1 $ values in $A[2..n]$ which are less than or equal to $a_1$, and if there are $m$ values in $A$ equal to $a_1$, then there are $r_1 - m$ elements in $A[2..n]$ which are less than $a_1$. Then we may repeat this process on $A[2..n]$ to continue finding the number of inversions of A. The above mentioned subtraction by m has the effect of ensuring that we always treat each nonunique item as lowest rank value which any element of that value can have.
    In our discussion of this chapter, we have split nonunique items to the right.
    Essentially, if we search for some element in an order--statistic tree, the first node with the value of our element will be the one with the lowest rank and all subsequent values will lie in the right subtree of the one with the lowest rank.
    Then, with all of this in mind, the solution to this problem is clear
    \newpage
    \begin{verbatim}
      COUNT-INVERSIONS(Array)
        let T be a new order-statistic tree
        n = Array.length

        for i = 1 to n      // n
          OS-insert(T,A[i]) // O(lg n)

        inversions = 0
        for j = 1 to n      // n
          x = RB-SEARCH(T.root,A[j]) // O(lg n)
          inversions = inversions + OS-RANK(T,x) - 1 // O(lg n)
          OS-delete(T,x) // O(lg n)
        return inversions
      // O(n lg n)
    \end{verbatim}

  \section*{15.1-1}
    Let $$T(n) = 1 + \sum_{j=0}^{n-1}T(j) \,\,\, \& \,\,\, T(0)=1 $$
    Then consider
    $$ T(n) - T(n-1) = \sum_{j=0}^{n-1}T(j) - \sum_{j=0}^{n-2}T(n) = T(n-1)$$
    So then $ T(n) - T(n-1) = T(n-1)$ which implies that $T(n) = 2T(n-1)$.
    Then, quite clearly, $T(n) = 2^{m}T(n-m)$ which implies that $T(n) = 2^{n}T(n-n) = 2^{n} T(0) = 2^n$

  \section*{15.1-2}
    Let $p = [1,10,1,1,24]$ and let $n = 5$.
    Our ``density" array, $ d = [1,5,0.3333,0.25,4.8]$ has a maximum which occurs at $2$.
    Then the most dense length that we can create is a rod of length $2$.
    Our greedy process will produce two rods of length $2$ and a rod of length $1$ for a total revenue of $21$.
    However, if we had not cut the rod at all, and left it as a rod of length $5$, we could have sold it for a revenue of $25$ which is higher than $21$, so our greedy strategy is not optimal.

  \newpage
  \section*{15.1-3}
    I shall deliver this algorithm in the form of pseudocode.
    \begin{verbatim}
      MEMOIZED-CUT-ROD-MODIFIED(p,n,c)
        let r[0..n] be a new array
        for i = 0 to n
          r[i] = -infinity
        return MEMOIZED-CUT-ROD-AUX-MODIFIED(p,n,c,r)

      MEMOZED-CUT-ROD-AUX-MODIFIED(p,n,c,r)
        // don't solve same problem twice
        if r[n] >= 0
          return r[n]

        // don't make money off of nothing
        if n == 0
          q = 0
        else
          q = -infinity
          for i = 1 to n-1
            // notice that we must modify the upper limit on our for loop
            // to avoid deducting c in the situation where we make no cuts
            q = max(q,p[i] + MEMOIZED-CUT-ROD-AUX-MODIFIED(p,n-i,c,r) - c)

        // nth step of for loop done outside the loop with no cuts
        r[n] = max(q,p[n])
        return r[n]
    \end{verbatim}
    Note that we still retain the property that we are never indebted by selling rods.
    The most important change in this algorithm is that the final step of the for loop cannot be executed with the other cases because, if it were, we would be penalized by $c$ without making any cuts.
    Moreover, the subtraction of x within the max inside the for loop reflects the penalty for making cuts. 
n


\end{document}
